\documentclass[acmtog,natbib=false]{acmart}
\AtBeginDocument{%
  \providecommand\BibTeX{{%
    Bib\TeX}}}

\RequirePackage[
datamodel=acmdatamodel,
style=acmnumeric, % use style=acmauthoryear for publications that require it
]{biblatex}
\addbibresource{references.bib}

\usepackage[T1]{fontenc}
\usepackage{xspace}
\usepackage[printonlyused]{acronym}

%% Rights management information.  This information is sent to you
%% when you complete the rights form.  These commands have SAMPLE
%% values in them; it is your responsibility as an author to replace
%% the commands and values with those provided to you when you
%% complete the rights form.
\setcopyright{acmlicensed}
\copyrightyear{2025}
\acmYear{2025}
\acmDOI{XXXXXXX.XXXXXXX}

%%
%% Submission ID.
%% Use this when submitting an article to a sponsored event. You'll
%% receive a unique submission ID from the organizers
%% of the event, and this ID should be used as the parameter to this command.
\acmSubmissionID{tbd}


%% useful macros
\newcommand{\todo}[1]{\textcolor{red}{#1}}
\newcommand{\code}[1]{\texttt{#1}\xspace}
\newcommand{\registered}[0]{\textsuperscript{\textregistered}\xspace}
\newcommand{\trademark}[0]{\texttrademark\xspace}

\begin{document}
\title{We need a Title for this Paper!}

\author{Dominik Adamski}
\email{dominik.adamski@amd.com}
\affiliation{%
  \institution{Advanced Micro Devices}
  \city{tbd}
  \state{tbd}
  \country{Poland}
}
\orcid{}

\author{Michael Klemm}
\email{michael.klemm@amd.com}
\orcid{0000-0002-8634-4634}
\affiliation{%
  \institution{Advanced Micro Devices GmbH}
  \city{Munich}
  \state{BY}
  \country{Germany}
}

\renewcommand{\shortauthors}{Adamski et al.}

\begin{abstract}s
Lorem ipsum dolor sit amet, consectetur adipisicing elit, sed do eiusmod
tempor incididunt ut labore et dolore magna aliqua. Ut enim ad minim veniam,
quis nostrud exercitation ullamco laboris nisi ut aliquip ex ea commodo
consequat. Duis aute irure dolor in reprehenderit in voluptate velit esse
cillum dolore eu fugiat nulla pariatur. Excepteur sint occaecat cupidatat non
proident, sunt in culpa qui officia deserunt mollit anim id est laborum.
Lorem ipsum dolor sit amet, consectetur adipisicing elit, sed do eiusmod
tempor incididunt ut labore et dolore magna aliqua. Ut enim ad minim veniam,
quis nostrud exercitation ullamco laboris nisi ut aliquip ex ea commodo
consequat. Duis aute irure dolor in reprehenderit in voluptate velit esse
cillum dolore eu fugiat nulla pariatur. Excepteur sint occaecat cupidatat non
proident, sunt in culpa qui officia deserunt mollit anim id est laborum.
\end{abstract}

%% generate via: https://dl.acm.org/ccs
\begin{CCSXML}
<ccs2012>
   <concept>
       <concept_id>10011007.10011006.10011041</concept_id>
       <concept_desc>Software and its engineering~Compilers</concept_desc>
       <concept_significance>500</concept_significance>
       </concept>
   <concept>
       <concept_id>10010147.10010169.10010175</concept_id>
       <concept_desc>Computing methodologies~Parallel programming languages</concept_desc>
       <concept_significance>500</concept_significance>
       </concept>
   <concept>
       <concept_id>10010520.10010521.10010542.10010546</concept_id>
       <concept_desc>Computer systems organization~Heterogeneous (hybrid) systems</concept_desc>
       <concept_significance>500</concept_significance>
       </concept>
 </ccs2012>
\end{CCSXML}
\ccsdesc[500]{Software and its engineering~Compilers}
\ccsdesc[500]{Computing methodologies~Parallel programming languages}
\ccsdesc[500]{Computer systems organization~Heterogeneous (hybrid) systems}

\keywords{LLVM, Flang, OpenMP, GPU, Accelerators}

\received{20 February 2007}
\received[revised]{12 March 2009}
\received[accepted]{5 June 2009}

\maketitle

\section{Introduction}
\label{sec:Introduction}
\Ac{GPU} is an acronym, and let's use a citation~\cite{AMD24}.

\section{LLVM Flang Compiler}
\label{sec:LLVMFlangCompiler}

\subsection{Fortran Compiler Pipeline}
\label{sec:FortranCompilerPipeline}
\todo{This is a copy of another paper; this needs to be reworded and extended.}
Figure~\ref{fig:FlangCompilerPhases} shows a (simplified) overview of the compiler pipeline of the AMD Next-Gen Fortran Compiler front-end.
The Prescanner \& Preprocessor stage produces a stream of characters of normalized Fortran source code. 
Preprocessor macros have been expanded, \code{INCLUDE} statements have been replaced with the included code, and unnecessary blank characters and comments have been removed.
Compiler directives such as \code{!\$dir} and the OpenMP directives introduced with \code{!\$omp} remain in the stream of characters.
The parser and semantic analysis stages then construct the parse tree of the processed Fortran code and annotate it with semantic information (e.g., data types).
In this representation, the OpenMP directives and their clauses are reflected as nodes in the parse tree.

\begin{figure}[t]
    \centering
    \includegraphics[width=\linewidth]{figures/flang_compiler_phases_overview.pdf}
    \caption{Simplified overview of the compiler phases in the AMD Next-Gen Fortran Compiler front-end.\label{fig:FlangCompilerPhases}}
\end{figure}

Through various lowering phases, the parse tree and the contained OpenMP directive structure are transformed into the \ac{MLIR} dialects for \ac{HLFIR} and \ac{FIR}.
\ac{HLFIR} provides additional high-level semantical information over \ac{FIR} (e.g., array operations that can be used to optimize Fortran array statements).
It also has additional information in the type system to represent Fortran attributes for (dummy) variables.
Both \ac{HLFIR} and \ac{FIR} contain explicit information about the OpenMP directives and clauses as well as their nesting structure.
For instance, an associated Fortran \code{DO} loop nest is associated with the corresponding OpenMP worksharing construct such as \code{LOOP}.
Finally, the \ac{MLIR} code is transformed into regular LLVM \ac{IR} and is passed to the compiler back-end for target code generation.

During lowering, the compiler outlines OpenMP \code{target} regions and creates kernel functions per region such that the back-end can generate \ac{GPU} code for those in addition to host code.
The original intermediate code is then replaced with boilerplate code to launch the created kernel functions.
This involves generated code to perform data mapping according to the OpenMP semantics and code to invoke the kernel on the GPU via the runtime libraries.

\subsection{OpenMP Code Generation}
\label{sec:OpenMPCodeGen}




\section{Conclusions}
\label{sec:Conclusions}

\section*{List of Acronyms}

\begin{acronym}[paper]
\acro{AST}[AST]{Abstract Syntax Tree}
\acro{API}[API]{application programming interface}
\acro{CCD}[CCD]{compute complex die}
\acro{CFD}[CFD]{computational fluid dynamics}
\acro{CU}[CU]{compute unit}
\acro{FIR}[FIR]{Fortran intermediate representation}
\acro{GPU}[GPU]{graphics processing unit}
\acro{HLFIR}[HLFIR]{high-level Fortran intermediate representation}
\acro{HPC}[HPC]{high-performance computing}
\acro{IR}[IR]{intermediate representation}
\acro{MLIR}[MLIR]{multi-level intermediate representation}
\acro{XCD}[XCD]{accelerator complex die}
\acro{WENO}[WENO]{weighted essentially non-oscillatory}
\acro{SDK}[SDK]{software development kit}
\acro{HIP}[HIP]{Heterogeneous-computing Interface for Portability}
\acro{IFA}[IFA]{isolated from above}
\acro{PFT}[PFT]{Pre-FIR Tree}
\end{acronym}

\section*{Acknowledgements}
AMD, the AMD Arrow logo, Instinct, Radeon, and EPYC, and combinations thereof are trademarks of Advanced Micro Devices, Inc.
Other product names used in this publication are for identification purposes only and may be trademarks of their respective companies.

\printbibliography

\end{document}
